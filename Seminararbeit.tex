% This is samplepaper.tex, a sample chapter demonstrating the
% LLNCS macro package for Springer Computer Science proceedings;
% Version 2.21 of 2022/01/12
%
\documentclass[runningheads]{llncs}
%
\usepackage[T1]{fontenc}
% T1 fonts will be used to generate the final print and online PDFs,
% so please use T1 fonts in your manuscript whenever possible.
% Other font encondings may result in incorrect characters.
%
\usepackage{graphicx}
% Used for displaying a sample figure. If possible, figure files should
% be included in EPS format.
%
% If you use the hyperref package, please uncomment the following two lines
% to display URLs in blue roman font according to Springer's eBook style:
%\usepackage{color}
%\renewcommand\UrlFont{\color{blue}\rmfamily}
%\urlstyle{rm}
%

%\usepackage[sectionbib,numbers,sort&compress]{natbib}
%\bibliographystyle{splncs04nat}

\begin{document}
%
\title{Usage of process mining for security in Process-Aware Information Systems}
%
\titlerunning{Process Mining for security in PAIS}
% If the paper title is too long for the running head, you can set
% an abbreviated paper title here
%
\author{Marlon Müller\inst{1}}
%
\authorrunning{M. Müller}
% First names are abbreviated in the running head.
% If there are more than two authors, 'et al.' is used.
%
\institute{
Technical University of Munich, Munich, Germany\\
\email{marlonbenedikt.mueller@tum.de}}
%
\maketitle              % typeset the header of the contribution
%
\begin{abstract}
Security in Process-Aware Information System (PAIS) is critical for almost every organisation and company, as a lack of security measures leads to vulnerabilities that can, if abused, cause severe fincancial damage.
The survey~\cite{Leitner2014273} by Leitner and Rinderle-Ma from 2013 analysed in a systematic literature review the researched security controls in PAIS\@. The authors identified that the usage
of Process Mining may be an emerging topic, so this paper provides a systematic literature review to analyze the advances in Process Mining research for security in PAIS and clusters the results
by the security goals that are protected as well as the Process Mining applications and concpets that can be used for security in PAIS\@. 

\keywords{Process Mining \and Security \and Process-Aware Information Systems \and Systematic literature review.}
\end{abstract}
%
%
%
\section{Introduction}\label{Introduction}
In 2018 76\% of businesses in Germany saw a significant risk for their business processes coming from cyberattacks on their information systems~\cite{Bsi} and in a report by the
Federal Bureau of Investigation from 2023 companies in the 
United States reported total damages of 37.5 billion US-Dollars due to cybercrime~\cite{fbi}. This shows the need for companies to protect their business processes from cyberattacks.\\
Therefore over the last years the research in security for Process-Aware Information Systems (PAIS) has increased and has produced a vast amount of research papers. The survey~\cite{Leitner2014273}
conducted a systematic literature review on security in PAIS and identified Process-Mining as an emerging technology that can be used to improve security in PAIS and predicted that
the usage of Process Mining for security purposes in PAIS will be a focus of future research.\\
In 2018, already 65\% of German businesses investigated log-files to identify security incidents based on the suspicion of a security incident and 33\% use log-files to systematically
identify security incidents without concrete suspicions~\cite{Bsi}.\\
Therefore Process Mining has the potential to be further enhanced to be used for security in Process-Aware Information Systems and was further researched in the last years. This paper
aims to update the survey by Leitner and Rinderle-Ma~\cite{Leitner2014273} in the field of Process Mining and to provide an overview over the recent advances in the usage of Process Mining for security purposes in PAIS since 2012. For this goal we will conduct a systematic literature
review to outline the possible applications of Process Mining for security in PAIS.\\
\\
To approach this goal the following research questions were targeted:
\begin{enumerate}
    \item How does Process Mining contribute to security in PAIS\@?
    \item Have the research challenges defined in~\cite{Leitner2014273} been addressed by recent research? 
\end{enumerate}

In Section~\ref{Fundamentals} we will first provide some defintions and explanaitions over the terms used in this paper. 
Section~\ref{Related} evaluates related works in this field and how this paper adds further contribution to current research.
Section~\ref{Methodology} outlines the different steps of the literature review consisting of literature search, literature selection, data extraction and the classification of security goals and security concepts.
The results of the literature review are described in Section~\ref{Results} and in Section~\ref{Conclusion} the results are discussed and the paper is concluded.

\section{Fundamentals}\label{Fundamentals}
Below definitons for the most relevant terms in this paper are provided.
\subsection{Process Mining}\label{Process Mining}
\cite{vanderAalst2016}
\subsection{Information Security}\label{Security}
TODO: IS definiton
\subsubsection{Security Goals}\label{goals_def}
Later in this paper the results will be clustered according to the security goals that are protected by the applications proposed by the reviewed papers (Section~\ref{Methodology} and~\ref{goals}). 
The clusters are chosen based on the security goals as described by Eckert in~\cite{Eckert}. While the CIA-Properties (Confidentiality, Integrity, Availabilty) are commonly agreed on as the 
primary security goals the other security goals (especially Privacy and Accountability) are acknowledged as valid security goals but their relevance to different use cases is discussed and 
not commonly agreed on. %Hier evtl andere Literatur noch nennen.
Nonetheless this six security to goals according to~\cite{Eckert} are chosen because clustering only by the CIA-Properties would not reflect the whole bandwith of security research in Process Mining
for PAIS while still allowing clear distinctions between them.
\paragraph{Confidentiality}
\paragraph{Integrity}
\paragraph{Availabilty}
\paragraph{Accountability}
\paragraph{Privacy}
\paragraph{Authenticity}
\subsection{Process-Aware Information Systems}\label{PAIS}
The term Process-Aware Information Systems is definded by Dumas et al.~\cite{Dumas20051} by combining the definitions of Information Systems (as PAIS are a special kind of Information Systems)
and business processes. Their understanding of a business process is a ``way for an organizational entity to organize work and resources (\ldots) to accomplish its aims''. On that foundation
PAIS are defined as ``a software system that manages and executes operational processes involving people, applications, and/or information sources on the basis of process models''.
Additionally to the formal defintion they marked that these process models are often represented using visual languages, like Petri-Net notations.\\
The main difference between a PAIS and a task-driven Information System (e.g.\ text editor or e-mail client) is described as the fact that task-driven applications are unaware of the process
they are used in and therefore can neither support nor restrict the user in the process execution.
\section{Related Work}\label{Related}
In this section the survey by Leitner and Rinderle-Ma~\cite{Leitner2014273}, that this paper aims to update, is briefly summarized and other related works are presented and explained how
this paper adds further contribution to the research in Process Mining for security in PAIS.\\
\subsection{Survey by Leitner and Rinderle-Ma from 2013}\label{basepaper}
The main objective of this paper is to update the systematic literature review conducted in~\cite{Leitner2014273} regarding the Process Mining security control and to provide an overview over the recent advances
in the usage of Process Mining for security purposes in PAIS until 2012. 
\subsubsection{Summary of the survey}\label{Summary}
In that survey the authors identified Process Mining as an emerging technology and gave a brief overview
over the security related research regarding Process Mining in PAIS\@. The authors assigned Process Mining to the action type Detection and placed it in the Change phase of the
process lifecycle.\\
In this survey the authors assigned Process Mining the action type Detection and placed it in the Change phase of the process lifecycle. They identified that the main usage of 
Process Mining for security purposes in PAIS is to examine the conformance of the process model with the actual process execution that could be derived from the event logs.
The derived model from the event logs can be used to detect inconsitencies and anomalious behaviour that could be an indicator for security incidents like fraud or intrusions.
Another use case the authors identified is to use event logs to validate the conformity of the process execution with the security policies of the company, like 
Role-Based Access Control (RBAC) models or data flow
policies. It was concluded, that Process Mining can be used to capture relevant information on data, ressources and task execution to find or address security issues and compliance
violations and their root causes.\\ 
\subsubsection{Research Challenges identified}\label{Challenges}
For security in PAIS in general (and not only regarding Process Mining) the authors identified the following research challenges:
\begin{enumerate}
    \item Agreement on Terminology and Controls
    \item Consistency with Related Fields and Concepts
    \item Measurement
    \item Testing
    \item Evaluation
    \item Detection Controls
    \item Reaction Controls
    \item Human Orientation
\end{enumerate}
\paragraph{Agreement on Terminology and Controls} The authors identified that the terminology used in the field of security in PAIS is not consistent and in some research no 
defintion of security or the protected security goals are provided.
\paragraph{Consistency with Related Fields and Concepts} Even though research in security in PAIS is an interdisciplinary field, it was identified that except for the NIST standard for
RBAC no standards or recoommendations are considered in the research.
\paragraph{Measurement} The authors could not identify any methods or metrics are used to evaluate the effectiveness of the security controls in PAIS, while in other security areas 
well developed standards, e.g.\ the ISO/IEC 27004 standard, exist.
\paragraph{Testing} Most of the security concepts in PAIS research are theoretical models and no method to test these models is provided.
\paragraph{Evaluation} In the examined research the evaluation of the security in PAIS centers on post-ex evaluation using Process Mining techniques and the authors stated that It
should be considered how to evaluate the security in PAIS at design or run time.
\paragraph{Detection Controls} Invastigations of possible security incidents does not happen at run time but the authors identified that it could be benficial to detect anomalies
at run time to then be able to react to them.
\paragraph{Reaction Controls} So far, reaction controls are focused on failure handling such as exception handling or process recovery. The authors identified that it could be beneficial
to react to identified security problems at design time.
\paragraph{Human Orientation} Human factors are not considered in the research instead it only focuses on the technological aspects of security in PAIS\@. The authors stressed that
humans are an important factor in business processes and that they also could be a security risk if they are psychologically manipulated during Social Engineering attacks. 
\subsection{Other Systematic Literature Reviews}\label{Other}
The surveys~\cite{Kelemen_2018} and~\cite{Silalahi20221} also conducted systematic literature reviews on the usage of Process Mining in security applications.


\section{Methodology}\label{Methodology}

\section{Results}\label{Results}
\subsection{How does Process Mining contribute to security in PAIS?}\label{Q1}
\subsubsection{Security goals}\label{goals}
\subsubsection{Security concepts}\label{concepts}
\subsection{Have the research challenges been addressed by recent research?}\label{Q2}

\section{Discussion and Conclusion}\label{Conclusion}

%
% ---- Bibliography ----
%
% BibTeX users should specify bibliography style 'splncs04'.
% References will then be sorted and formatted in the correct style.
% TODO: Change to splncs04nat
% \bibliographystyle{splncs04}
% \bibliography{mybibliography}
%
\bibliographystyle{plain}
\bibliography{Seminararbeit}{}
\end{document}
